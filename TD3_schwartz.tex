\documentclass[11pt]{amsart}

\usepackage[french]{babel}
\usepackage[utf8]{inputenc}
\usepackage[T1]{fontenc}
\usepackage{bbm}
\usepackage{graphicx}
\usepackage{amsfonts}
\usepackage{amssymb}
\usepackage{amsthm}
\usepackage{amsmath}
\usepackage{xcolor}
\usepackage{mathtools}
\usepackage{bm}
\usepackage{braket}
\usepackage{dsfont}
\usepackage{upgreek}
\usepackage[norelsize,french,ruled]{algorithm2e}
\usepackage{multirow}
\usepackage{enumitem}
\usepackage{tikz}
%\addtolength{\headheight}{1cm}



\renewcommand{\refname}{R\'ef\'erences}

\oddsidemargin = 0pt \evensidemargin = 0pt \marginparwidth = 1in%\marginparsep = 0pt 
\leftmargin = 1.25in \topmargin = 0pt
\headsep = 10pt \topskip = 0pt \footskip = 0.25in 
\textheight = 22cm \textwidth = 6.5in

\newtheoremstyle{exercice}
{3pt} %Space  above
{5pt} %Space below
{} %Body font
{} %Indent amount, empty no indent % \parindent=paragraph
{\bfseries} %Theorem head font
{.} %Ponctuation after theorem head
{0.5em} %Space after theorem head. \newline=linebreak
{}%Theorem head spec {can be left empty meaning normal

\newtheoremstyle{question}{2pt}{0pt}{}{}{\bfseries}{}{0pt}{#2) #3}

\newtheorem{theorem}{Th\'eor\`eme}
\newtheorem{lemma}[theorem]{Lemme}
\newtheorem{proposition}[theorem]{Proposition}
\newtheorem{corollary}[theorem]{Corollaire}
\theoremstyle{remark}
\newtheorem{remark}{Remarque}
\theoremstyle{exercice}
\newtheorem{exo}{Exercice}
\newtheorem{ex}{Exemple}

\newcommand\blfootnote[1]{%
  \begingroup
  \renewcommand\thefootnote{}\footnote{#1}%
  \addtocounter{footnote}{-1}%
  \endgroup
}

\newcommand{\titre}[3] 
% #1: titre #2: classe #3: ann\'ee
{
\noindent\parbox{7cm}
{
\noindent \textbf{#2}\\
#3
}
\hfill
%\parbox{5cm}
{ 
Année 2024--2025}

\vspace{0.5 cm} 

\begin{center}\rule{\textwidth}{0.2mm}\end{center}
\vspace{0.2cm}
\begin{center}{\large{\textsc{#1}}}\end{center}
\begin{center}\rule{\textwidth}{0.2mm}\end{center}

\vspace{0.5cm}
}

\newcommand{\R}{\mathbb{R}}
\newcommand{\C}{\mathbb{C}}
\newcommand{\N}{\mathbb{N}}
\newcommand{\Z}{\mathbb{Z}}
\newcommand{\Q}{\mathbb{Q}}
\newcommand{\F}{\mathbb{F}}
\newcommand{\K}{\mathbb{K}}
\newcommand{\un}{(u_n)_{n\in \N}}
\newcommand{\an}{(a_n)_{n\in \N}}
\newcommand{\Sn}{(S_n)_{n\in \N}}
\newcommand{\Rn}{(R_n)_{n\in \N}}
\newcommand{\se}{\sum_{n\ge 0} u_n}

\newcommand{\eps}{\varepsilon}
\newcommand{\defi}[1]{{\bf #1}}

\DeclareMathOperator{\card}{card}
\DeclareMathOperator{\re}{Re}
\DeclareMathOperator{\im}{Im}

\setlist[itemize,1]{label=\textbullet}

\makeatletter
\@addtoreset{section}{part}
\makeatother

%\makeatletter
%\renewcommand\part{\@startsection{part}{2}%
%  \z@{.5\linespacing\@plus.7\linespacing}{-.5em}%
%  {\normalfont\scshape}}
%\makeatother

\makeatletter
\def\part{\@startsection{part}{0}%
  \z@{\linespacing\@plus\linespacing}{.5\linespacing}%
  {\normalfont\scshape\raggedright}}
\makeatother

%%%%%%%%%%%%%%%%%%%%%%%%%

\begin{document}

\titre{TD 3 : classe de Schwartz, transformée de Fourier, distributions tempérées \blfootnote{\textbf{Auteur :} Simona ROTA NODARI \\ \texttt{simona.rotanodari@univ-cotedazur.fr}}}{Université Côte d'Azur}{M1 Analyse de Fourier et distributions}

%%%%%%%%%%%%%%%%%%%%%%%%%%%%%%%%%%%%%%%%%%%%%%%%%%%%%%%%%%%%%%%%

% Dans la suite, nous noterons $m_d$ la mesure de Lebesgue $\lambda$ sur $\R^d$ divisée par $(2\pi)^{d/2}$ et $m$ la mesure $m_1$. De plus, pour tout $p\in[1,\infty[$, nous noterons $L^p(\R^d):=L^p(\R^d,m_d)$ avec norme $\|f\|_p=(\int_{\R^d}|f(x)|^p\,dm_d(x)))^{1/p}$.

% Soit $f\in L^1(\R^d)$. La \defi{transformée de Fourier} de $f$ est la fonction 
% \begin{equation*}\label{defFourier}
%   \mathcal F(f)(t)=\hat f(t)=\int_{\R^d} f(x)e^{-ix\cdot t}\,dm_d(x)=\frac{1}{(2\pi)^{d/2}}\int_{\R^d} f(x)e^{-ix\cdot t}\,d\lambda_d(x).
% \end{equation*}



\exo{
  Montrer que $\mathcal C_c^\infty(\R)$ est un sous-espace dense de $\mathcal S(\R)$
}

\exo{{\bf Formule de Leibniz} Soient $f,g\in \mathcal C^{\infty}(R^d)$ et $\alpha=(\alpha_1,\ldots,\alpha_d)\in \N^d$. Montrer que
\begin{equation*}
  \partial^{\alpha}(fg)=\sum_{\mathclap{\substack{\beta\in \N^d\\\beta\le \alpha}}}\binom{\alpha}{\beta}\partial^\beta f\partial^{\alpha-\beta}g 
\end{equation*}
avec $\binom{\alpha}{\beta}=\frac{\alpha !}{\beta !(\alpha-\beta)!}$, $\alpha !=\alpha_1 ! \cdots \alpha_d !$ et où $\beta \le \alpha$ signifie $\alpha_i\le \beta_i$ pour tout $i=1,\ldots,d$. \emph{Indication : utiliser une preuve par induction sur $|\alpha|=\sum_{i=1}^{d}\alpha_i$.}
}

\exo{ Soient $\phi,\psi\in \mathcal S(\R^d)$. Montrer que $\phi\star\psi\in \mathcal S(\R^d)$.

}

\exo{
  Soit $a>0$. Pour tout $x\in \R$, on pose $g_a(x)=e^{-\frac{a }{2}\xi^2}$. Montrer que $g_a\in \mathcal S(\R)$ et que sa transformée de Fourier est donnée par
  \begin{equation*}
    \mathcal F g_a(\xi)= \frac{1}{\sqrt{a}}e^{-\frac{x^2}{2a}}
  \end{equation*}
  pour tout $\xi\in \R$. 
  
  \emph{On pourra utiliser le fait que $\int_\R e^{-a x^2}\,dx=\sqrt{\frac{\pi}{a}}$.}
}

\exo{Soit $\phi\in \mathcal S(\R^d)$ et $f\in \mathcal C^{\infty}(\R^d)$ avec toutes les dérivées à croissance polynomiale (\emph{i.e.} pour tout $\alpha\in \N^d$, il existe $n\in \N$ tel que $\partial^{\alpha}f = O_{|x|\to \infty}(|x|^n)$). Montrer, à l'aide de la formule de Leibniz, que $f\phi\in \mathcal S(\R^d)$.}

\exo{
  Soit $x_0\in \R$ et $\delta_{x_0}$ la distribution tempérée donnée par $\langle\delta_{x_0}, \phi\rangle=\phi(x_0)$ pour tout $\phi\in \mathcal S(\R)$. Pour tout $m\in \N^*$, calculer $\partial_x^{m} \delta_{x_0}$ et montrer qu'il s'agit d'une distribution tempérée.
}

\exo{{\bf Valeur principale de $\tfrac{1}{x}$.} On considère la forme linéaire définie par 
\begin{equation*}
  \mathrm{vp}\frac{1}{x} : \phi\in \mathcal S(\R)\mapsto \langle \mathrm{vp}\frac{1}{x},\phi \rangle =\lim_{\varepsilon\to 0^+}\int_{|x|>\varepsilon}\frac{\phi(x)}{x}\,dx.
\end{equation*}
\begin{enumerate}
  \item Montrer que $\mathrm{vp}\frac{1}{x}$ est dans $\mathcal S'(\R)$.
  \item Montrer que $f: x\in \R\to \ln|x|$ est un élément de $\mathcal S'(\R)$ (en utilisant l'identification $f\to T_f$) et sa dérivée au sens des distributions est donnée par $\mathrm{vp}\frac{1}{x}$.
  \item Pour tout $\varepsilon>0$, on pose $H_{\varepsilon}(x)=e^{-\varepsilon x} H(x)$ avec $H$ la fonction d'Heaviside $H(x)=\mathbbm{1}_{\R_+}(x)$.
  \begin{enumerate}
    \item Montrer que $H_{\varepsilon}$ converge vers $H$ dans $\mathcal S'(\R)$ lorsque $\varepsilon \to 0$. 
    %En déduire que $\mathcal F H_{\varepsilon}$ converge vers $\mathcal F H$ dans $\mathcal S'(\R)$ lorsque $\varepsilon \to 0$.
    \item Calculer la transformée de Fourier de $H_\varepsilon$ et en déduire la transformée de Fourier de $H$.
    \item À l'aide de la question précedente, calculer la transformée de Fourier de
    $\mathrm{vp}\frac{1}{x}$.
  \end{enumerate}
\end{enumerate}
}



\end{document}
