\documentclass[11pt]{amsart}

\usepackage[french]{babel}
\usepackage[utf8]{inputenc}
\usepackage[T1]{fontenc}
\usepackage{bbm}
\usepackage{graphicx}
\usepackage{amsfonts}
\usepackage{amssymb}
\usepackage{amsthm}
\usepackage{amsmath}
\usepackage{xcolor}
\usepackage{mathtools}
\usepackage{bm}
\usepackage{braket}
\usepackage{dsfont}
\usepackage{upgreek}
\usepackage[norelsize,french,ruled]{algorithm2e}
\usepackage{multirow}
\usepackage{enumitem}
\usepackage{tikz}
%\addtolength{\headheight}{1cm}



\renewcommand{\refname}{R\'ef\'erences}

\oddsidemargin = 0pt \evensidemargin = 0pt \marginparwidth = 1in%\marginparsep = 0pt 
\leftmargin = 1.25in \topmargin = 0pt
\headsep = 10pt \topskip = 0pt \footskip = 0.25in 
\textheight = 22cm \textwidth = 6.5in

\newtheoremstyle{exercice}
{3pt} %Space  above
{5pt} %Space below
{} %Body font
{} %Indent amount, empty no indent % \parindent=paragraph
{\bfseries} %Theorem head font
{.} %Ponctuation after theorem head
{0.5em} %Space after theorem head. \newline=linebreak
{}%Theorem head spec {can be left empty meaning normal

\newtheoremstyle{question}{2pt}{0pt}{}{}{\bfseries}{}{0pt}{#2) #3}

\newtheorem{theorem}{Th\'eor\`eme}
\newtheorem{lemma}[theorem]{Lemme}
\newtheorem{proposition}[theorem]{Proposition}
\newtheorem{corollary}[theorem]{Corollaire}
\theoremstyle{remark}
\newtheorem{remark}{Remarque}
\theoremstyle{exercice}
\newtheorem{exo}{Exercice}
\newtheorem{ex}{Exemple}

\newcommand\blfootnote[1]{%
  \begingroup
  \renewcommand\thefootnote{}\footnote{#1}%
  \addtocounter{footnote}{-1}%
  \endgroup
}

\newcommand{\titre}[3] 
% #1: titre #2: classe #3: ann\'ee
{
\noindent\parbox{7cm}
{
\noindent \textbf{#2}\\
#3
}
\hfill
%\parbox{5cm}
{ 
Année 2024--2025}

\vspace{0.5 cm} 

\begin{center}\rule{\textwidth}{0.2mm}\end{center}
\vspace{0.2cm}
\begin{center}{\large{\textsc{#1}}}\end{center}
\begin{center}\rule{\textwidth}{0.2mm}\end{center}

\vspace{0.5cm}
}

\newcommand{\R}{\mathbb{R}}
\newcommand{\C}{\mathbb{C}}
\newcommand{\N}{\mathbb{N}}
\newcommand{\Z}{\mathbb{Z}}
\newcommand{\Q}{\mathbb{Q}}
\newcommand{\F}{\mathbb{F}}
\newcommand{\K}{\mathbb{K}}
\newcommand{\un}{(u_n)_{n\in \N}}
\newcommand{\an}{(a_n)_{n\in \N}}
\newcommand{\Sn}{(S_n)_{n\in \N}}
\newcommand{\Rn}{(R_n)_{n\in \N}}
\newcommand{\se}{\sum_{n\ge 0} u_n}

\newcommand{\eps}{\varepsilon}
\newcommand{\defi}[1]{{\bf #1}}

\DeclareMathOperator{\card}{card}
\DeclareMathOperator{\re}{Re}
\DeclareMathOperator{\im}{Im}

\setlist[itemize,1]{label=\textbullet}

\makeatletter
\@addtoreset{section}{part}
\makeatother

%\makeatletter
%\renewcommand\part{\@startsection{part}{2}%
%  \z@{.5\linespacing\@plus.7\linespacing}{-.5em}%
%  {\normalfont\scshape}}
%\makeatother

\makeatletter
\def\part{\@startsection{part}{0}%
  \z@{\linespacing\@plus\linespacing}{.5\linespacing}%
  {\normalfont\scshape\raggedright}}
\makeatother

%%%%%%%%%%%%%%%%%%%%%%%%%

\begin{document}

\titre{TD 2 : transformée de Fourier \blfootnote{\textbf{Auteur :} Simona ROTA NODARI \\ \texttt{simona.rotanodari@univ-cotedazur.fr}}}{Université Côte d'Azur}{M1 Analyse de Fourier et distributions}

%%%%%%%%%%%%%%%%%%%%%%%%%%%%%%%%%%%%%%%%%%%%%%%%%%%%%%%%%%%%%%%%

Dans la suite, nous noterons $m_d$ la mesure de Lebesgue $\lambda$ sur $\R^d$ divisée par $(2\pi)^{d/2}$ et $m$ la mesure $m_1$. De plus, pour tout $p\in[1,\infty[$, nous noterons $L^p(\R^d):=L^p(\R^d,m_d)$ avec norme $\|f\|_p=(\int_{\R^d}|f(x)|^p\,dm_d(x)))^{1/p}$.

Soit $f\in L^1(\R^d)$. La \defi{transformée de Fourier} de $f$ est la fonction 
\begin{equation*}\label{defFourier}
  \mathcal F(f)(t)=\hat f(t)=\int_{\R^d} f(x)e^{-ix\cdot t}\,dm_d(x)=\frac{1}{(2\pi)^{d/2}}\int_{\R^d} f(x)e^{-ix\cdot t}\,d\lambda_d(x).
\end{equation*}

\exo{
  Soient $f\in L^1(\R^d)$ et $\alpha,\lambda\in \R$. Montrer les propriétés suivantes.
    \begin{enumerate}
        \item Si $g(x)=f(x)e^{i\alpha x}$, alors $\hat g(t)=\hat f(t-\alpha)$.
        \item Si $g(x)=f(x-\alpha)$, alors $\hat g(t)=\hat f(t)e^{-i\alpha t}$.
        %\item Si $g\in L^1(\R^n)$, alors $\widehat{f\star g}(t)=\hat f(t)\hat g(t)$.
        \item Si $g(x)=\overline{f(-x)}$, alors $\hat g(t)=\overline{\hat {f}(t)}$.
        \item Si $g(x)=f\left(\frac{x}{\lambda}\right)$ and $\lambda>0$, alors $\hat g(t)=\lambda^n\hat f(\lambda t)$.
        %\item Si $g_k(x)=-ix_kf(x)$ et $g_k\in L^1(\R^n)$ pour tout $k=1,\ldots,n$, alors $\hat f$ est de classe $\mathcal C^1$ et $\partial_{t_k}\hat f(t)=\hat g_k(t)$. 
    \end{enumerate}
}

\exo{Calculer la transformée de Fourier des fonctions suivantes.
\begin{enumerate}
  \item $f(x)=\chi_{[-a,a]}(x)$, $x\in \R$ et $a>0$.
  \item $f(x)=\chi_{[a,b]}(x)$, $x\in \R$ et $a, b\in \R$, $a<b$.
  \item \begin{equation*}
    f(x)=\left\{
      \begin{aligned}
        &1+x& -1\le x\le 0\\
        &1-x& 0\le x\le 1\\
        &0& |x|>1
      \end{aligned}
    \right..
  \end{equation*}
  Combien vaut $\hat f(0)$?
\end{enumerate}
}

\exo{
  Soit $f\in L^1(\R)$ et $\alpha\in \R$. Montrer que la fonction $\varphi(t)=\hat f(t)\cos(\alpha t)$ est la transformée de Fourier d'une fonction de $L^1(\R)$ que l'on explicitera en fonction de $f$ et $\alpha$. Même question pour $\psi(t)=\hat f(t)\sin(\alpha t)$.
}

\exo{
  Soit $f:\R\to\R$ une fonction de classe $\mathcal C^2(\R)$ telle que $f, f', f''\in L^1(\R)$. Montrer que $\hat f\in L^1(\R)$.
}

\exo{
  \begin{enumerate}
    \item En utilisant la transformée de Fourier, montrer qu'il n'existe pas de fonction $g\in L^1(\R^d)$ telle que $f\star g=f$ pour tout $f\in L^1(\R^d)$. 
    \item Resoudre dans $L^1(\R^d)$ l'équation $f\star f=f$.
  \end{enumerate}
}

\exo{ Pour $\alpha>0$, on pose $f_\alpha(x)=e^{-\alpha|x|}$ pour tout $x\in \R$.
\begin{enumerate}
\item Calculer la transformée de Fourier de $f_\alpha$.
\item À l'aide du théorème d'inversion, en déduire la transformée de Fourier de $x\mapsto \frac{1}{\alpha^2+x^2}$.
\item Calculer $f_\alpha\star f_\alpha$ et en déduire la transformée de Fourier de $x\mapsto \frac{1}{(\alpha^2+x^2)^2}$.
\item Déterminer la transformée de Fourier de $x\mapsto \frac{x}{(1+x^2)^2}$.
\end{enumerate}
}

\exo{ Soient $X$ et $Y$ deux espaces métriques, $Y$ complet. Soient $X_0$ un sous-ensemble dense de $X$ et $f:X_0\to Y$ une application uniformément continue.
\begin{enumerate}
    \item Montrer que l'application $f$ se prolonge en une unique fonction uniformément continue $\tilde f:X\to Y$.
    \item Montrer que si $f(X_0)$ est un sous-ensemble dense de $Y$, avec $f$ isométrique et $X$ également complet,
	  alors $\tilde f$ est surjective de $X$ dans $Y$.
\end{enumerate}
}

\exo{
  À l'aide de la trasformée de Fourier, calculer les intégrales suivantes.
  \begin{enumerate}
    \item   $\int_{\R}\frac{\sin(t)^2}{t^2}\,dt$.
    \item $\int_{\R}\frac{1}{(a^2+x^2)(b^2+x^2)}\,dx$, $a,b>0$.
  \end{enumerate}

}

\end{document}
