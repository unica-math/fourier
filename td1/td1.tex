\documentclass[11pt]{amsart}

\usepackage[french]{babel}
\usepackage[utf8]{inputenc}
\usepackage[T1]{fontenc}
\usepackage{bbm}
\usepackage{graphicx}
\usepackage{amsfonts}
\usepackage{amssymb}
\usepackage{amsthm}
\usepackage{amsmath}
\usepackage{xcolor}
\usepackage{mathtools}
\usepackage{bm}
\usepackage{braket}
\usepackage{dsfont}
\usepackage{upgreek}
\usepackage[norelsize,french,ruled]{algorithm2e}
\usepackage{multirow}
\usepackage{enumitem}
\usepackage{tikz}
%\addtolength{\headheight}{1cm}



\renewcommand{\refname}{R\'ef\'erences}

\oddsidemargin = 0pt \evensidemargin = 0pt \marginparwidth = 1in%\marginparsep = 0pt 
\leftmargin = 1.25in \topmargin = 0pt
\headsep = 10pt \topskip = 0pt \footskip = 0.25in 
\textheight = 22cm \textwidth = 6.5in

\newtheoremstyle{exercice}
{3pt} %Space  above
{5pt} %Space below
{} %Body font
{} %Indent amount, empty no indent % \parindent=paragraph
{\bfseries} %Theorem head font
{.} %Ponctuation after theorem head
{0.5em} %Space after theorem head. \newline=linebreak
{}%Theorem head spec {can be left empty meaning normal

\newtheoremstyle{question}{2pt}{0pt}{}{}{\bfseries}{}{0pt}{#2) #3}

\newtheorem{theorem}{Th\'eor\`eme}
\newtheorem{lemma}[theorem]{Lemme}
\newtheorem{proposition}[theorem]{Proposition}
\newtheorem{corollary}[theorem]{Corollaire}
\theoremstyle{remark}
\newtheorem{remark}{Remarque}
\theoremstyle{exercice}
\newtheorem{exo}{Exercice}
\newtheorem{ex}{Exemple}

\newcommand\blfootnote[1]{%
  \begingroup
  \renewcommand\thefootnote{}\footnote{#1}%
  \addtocounter{footnote}{-1}%
  \endgroup
}

\newcommand{\titre}[3] 
% #1: titre #2: classe #3: ann\'ee
{
\noindent\parbox{7cm}
{
\noindent \textbf{#2}\\
#3
}
\hfill
%\parbox{5cm}
{ 
Année 2024--2025}

\vspace{0.5 cm} 

\begin{center}\rule{\textwidth}{0.2mm}\end{center}
\vspace{0.2cm}
\begin{center}{\large{\textsc{#1}}}\end{center}
\begin{center}\rule{\textwidth}{0.2mm}\end{center}

\vspace{0.5cm}
}

\newcommand{\R}{\mathbb{R}}
\newcommand{\C}{\mathbb{C}}
\newcommand{\N}{\mathbb{N}}
\newcommand{\Z}{\mathbb{Z}}
\newcommand{\Q}{\mathbb{Q}}
\newcommand{\F}{\mathbb{F}}
\newcommand{\K}{\mathbb{K}}
\newcommand{\un}{(u_n)_{n\in \N}}
\newcommand{\an}{(a_n)_{n\in \N}}
\newcommand{\Sn}{(S_n)_{n\in \N}}
\newcommand{\Rn}{(R_n)_{n\in \N}}
\newcommand{\se}{\sum_{n\ge 0} u_n}

\newcommand{\eps}{\varepsilon}
\newcommand{\defi}[1]{{\bf #1}}

\DeclareMathOperator{\card}{card}
\DeclareMathOperator{\re}{Re}
\DeclareMathOperator{\im}{Im}

\setlist[itemize,1]{label=\textbullet}

\makeatletter
\@addtoreset{section}{part}
\makeatother

%\makeatletter
%\renewcommand\part{\@startsection{part}{2}%
%  \z@{.5\linespacing\@plus.7\linespacing}{-.5em}%
%  {\normalfont\scshape}}
%\makeatother

\makeatletter
\def\part{\@startsection{part}{0}%
  \z@{\linespacing\@plus\linespacing}{.5\linespacing}%
  {\normalfont\scshape\raggedright}}
\makeatother

%%%%%%%%%%%%%%%%%%%%%%%%%

\begin{document}

\titre{TD 1 : espaces $L^p$ \blfootnote{\textbf{Auteur :} Simona ROTA NODARI \\ \texttt{simona.rotanodari@univ-cotedazur.fr}}}{Université Côte d'Azur}{M1 Analyse de Fourier et distributions}

%%%%%%%%%%%%%%%%%%%%%%%%%%%%%%%%%%%%%%%%%%%%%%%%%%%%%%%%%%%%%%%%



% \exo{(Inégalité de Tchebychev)\\
% Soit $f\in L^p(E,\mu;\K)$ avec $1\leq p < \infty$. Montrer l'in\'egalit\'e suivante :
% $$
% \forall \varepsilon>0, \quad  \mu(\{|f|>\varepsilon\})\leq \frac{||f||^p_p}{\varepsilon^p}.
% $$
% }

% \exo{
% Soit l'espace mesur\'e $(\N,\mathcal P(\N),\mu)$ o\`u $\mu$ est la mesure de comptage. On pose 
% $\ell^p=L^p(\N,\mu;\C)$.
% \begin{enumerate}
% \item Montrer  que si $1\le p\leq q \leq +\infty$, alors $\ell^p \subset \ell^q$.
% \item Montrer que l'injection canonique $j\colon \ell^p\rightarrow \ell^q$ est une application lin\'eaire continue
% et calculer sa norme.
% \item Montrer que si $1\le p< q\le +\infty$ alors l'inclusion $\ell^p \subset \ell^q$ est stricte.
% \item Montrer que si $1\le p<+\infty$ alors le sous-espace vectoriel $V$ engendré par les suites $e{(k)}$, $e{(k)}$ étant la suite qui vaut $1$ au $k$-ième terme et $0$ sinon, est dense dans $\ell^p$. En déduire que $\ell^p$ est séparable pour $1\le p<+\infty$.
% \end{enumerate}
% }

\exo{
  Soit $(E,\mathcal E)$ un espace mesurable et soit $f:E\to \R_+$ une fonction mesurable positive. Montrer qu'il existe une suite de fonctions étagées mesurables $(s_n)_n$ telle que
  \begin{enumerate}
    \item $0\le s_1\le s_2\le \ldots \le f$, 
    \item $s_n$ converge simplement vers $f$ sur $E$.
  \end{enumerate}
}

\exo{(Inégalité de Hölder généralisée)\\
Soient $p,q\in [1,+\infty]$ et $r$ défini par $\frac{1}{r}=\frac{1}{p}+\frac{1}{q}$. Montrer que pour $f\in L^p(E,\mu;\K)$ et $g \in L^q(E,\mu;\K)$ on a :
$$
||fg||_r\le ||f||_p||g||_q.
$$
}

\exo{
Soit $(E ,\mathcal E,\mu)$ un espace mesur\'e tel que $\mu(E)<\infty$. Soient $1\leq p\leq q \leq +\infty$.
Montrer que l'injection canonique $j\colon L^q\rightarrow L^p$ est une application lin\'eaire continue
et calculer sa norme.
}

% \exo{Soit $f\in L^r(E,\mu;\K)$ pour $r<+\infty$. Montrer que
% $$
% \lim_{p\to +\infty}||f||_p=||f||_{\infty}.
% $$
% \emph{Indication : Utiliser l'inégalité de Tchebychev.}
% }

\exo{
\begin{enumerate} 
\item Soit $E$ un espace de Banach. On suppose qu'il existe une famille $(\mathcal O_i)_{i\in I}$ telle que :
\begin{enumerate}
\item pour tout $i\in I$, $\mathcal O_i$ est un ouvert non vide de $E$,
\item $\mathcal O_i\cap \mathcal O_j=\emptyset$ si $i\neq j$,
\item $I$ n'est pas dénombrable.
\end{enumerate}
Montrer que $E$ n'est pas séparable. (\emph{Indication : raisonner par l'absurde}).
\item Pour tout $x\in \R^d$, on pose $f_x=\chi_{\mathcal B(x,1)}$ où $\mathcal B(x,1)\subset \R^d$ est la boule fermée de centre $x$ et de rayon $1$. En utilisant la famille d'ouverts $(\mathcal O_x)_{x\in \R^d}$ avec 
$$
\mathcal O_x=\left\{f\in L^\infty(\R^d,\lambda_d), ||f-f_x||_{\infty}<\frac{1}{2}\right\},
$$
montrer que $L^\infty(\R^d,\lambda_d)$ n'est pas séparable.
\end{enumerate}
}

\exo{ Soit $1\le p <\infty$. Soit $(f_n)_{n\in\N}$ une suite dans $L^p(E,\mu;\K)$ qui converge
simplement presque partout vers une fonction $f$ de $L^p(E,\mu;\K)$.
Montrer que la suite $(f_n)_{n\in\N}$ converge vers $f$ dans $L^p(E,\mu;\K)$ si et seulement si
$\lim_{n\to\infty} ||f_n||_p = ||f||_p$.
}

\exo{
Pour une fonction $f\colon\R\rightarrow \C$ et pour $h\in\R$, on d\'efinit la 
fonction $\tau_h f$ par $(\tau_h f)(x)= f(x+h)$.
\begin{enumerate}
\item Soit $p\in[1,\infty[$. 
\begin{enumerate}
\item Montrer que pour tout $h\in \R$, $\tau_h$ d\'efinit une isom\'etrie de $L^p=L^p(\R,\lambda)$.
\item Soient  $g\in C_c(\R)$  une fonction continue \`a support compact. 
Montrer que 
$$
\forall \varepsilon>0,\ \exists \delta>0,\ \forall (a,b)\in \R^2,\ |a-b|<\delta \Rightarrow ||\tau_a g - \tau_b g||_p<\varepsilon.
$$
\item En utilisant la densit\'e de $C_c(\R)$ dans $L^p$,
montrer que si $f\in L^p$, on a   $\lim_{h\rightarrow 0}||\tau_h f - f||_p=0$.
\end{enumerate}
%\item Donner des contre-exemples simples qui montrent que les r\'esultats (b) et (c) ne sont pas valables pour $L^\infty$.
\item Donner un contre-exemple simple qui montre que le r\'esultat (c) n'est pas valable pour $L^\infty$.
\end{enumerate}
}

\exo{ Soient $1\le p<+\infty$ et $1<q\le +\infty$ tels que $\frac{1}{p}+\frac{1}{q}=1$. On suppose $f\in L^p(\R,\lambda)$ et $g\in L^q(\R,\lambda)$ avec $\lambda$ la mesure de Lebesgue. 
\begin{enumerate}
\item Montrer que le produit de convolution de $f$ et $g$ est une fonction bornée sur $\R$ qui vérifie
$$
\|f\star g\|_{\infty}\le \|f\|_p\|g\|_q.
$$
\item Montrer que $f\star g$ est uniformément continue sur $\R$. 

\end{enumerate}
}

\exo{ Soient $f\in L^p(\R,\lambda)$ et $g\in L^q(\R,\lambda)$ avec $1\le p, q\le +\infty$ tels que
$1/p + 1/q = 1 + 1/r$, avec $1 \leq r \leq +\infty$, et $\lambda$ la mesure de Lebesgue.
Montrer que pour presque tout $x\in \R$, la fonction $y\mapsto f(x-y)g(y)$ est intégrable sur $\R$ et que le produit de convolution de $f$ et $g$ défini par
$$
f\star g(x)=\int_{\R} f(x-y)g(y)\,dy
$$
est commutatif, appartient à $L^r(\R,\lambda)$ et vérifie
$$
\|f\star g\|_r\le \|f\|_p\|g\|_q
$$
}

\exo{Soient $\alpha,\beta\in \R$ tels que $\alpha<\beta$. On pose $f=\mathds{1}_{[-\alpha, \alpha]}$ et $g=\mathds{1}_{[-\beta,\beta]}$.
\begin{enumerate}
\item Montrer que le produit de convolution de $f$ et $g$ est bien défini.
\item Calculer $f\star g$ pour tout $x\in \R$. 
\item Étudier la régularité de $f$, $g$ et $f\star g$.
\end{enumerate}
}

\exo{
  \begin{enumerate}
    \item Soient $f\in L^1(\R^d,\lambda)$ et $g\in L^p(\R^d,\lambda)$ avec $p\in [1,\infty[$. Montrer que 
    \begin{equation*}
      \mathrm{supp}(f\star g)\subset\overline{\mathrm{supp}\,f+\mathrm{supp}\,g}.
    \end{equation*}
    \item Soient $f \in \mathcal{C}_c(\R^d)$ et $g\in L^1_{\mathrm{loc}}(\R^d,\lambda)$. Montrer que $f \star g$ est $\mathcal{C}(\R^d)$.
    \item Soient $f \in \mathcal{C}^\infty_c(\R^d)$ et $g\in L^1_{\mathrm{loc}}(\R^d,\lambda)$. Montrer que $f \star g$ est $\mathcal{C}^\infty(\R^d)$.
    \item Soit $\rho\in \mathcal C_c^\infty(\R^d)$ avec $\mathrm{supp}\,\rho\subset \overline{B(0,1)}$, $\rho\ge 0$ sur $\R^d$ et $\|\rho\|_1=\int_{\R^d}\rho\,d\lambda>0$. On pose $$\rho_n(x)=Cn^d\rho(nx)$$ avec $C=\|\rho\|_1^{-1}$.
      \begin{enumerate}
        \item Soit $f\in \mathcal C(\R^d)$. Montrer que $\rho_n\star f$ converge uniformément vers $f$ sur tout compact de $\R^d$.
        \item Soit $f\in \mathcal C_c(\R^d)$. Montrer que $\lim_{n\to \infty}\|\rho_n\star f -f\|_p=0$.
        \item En déduire que $\mathcal C_c^\infty(\R^d)$ est dense dans $L^p(\R^d,\lambda)$ pour $p\in [1,\infty[$. \emph{Indication: on pourra utiliser la densité de $\mathcal C_c(\R^d)$ dans $L^p(\R^d,\lambda)$}.
      \end{enumerate}
  \end{enumerate}
}

% \exo{ Soient $f,g\in L^1(\R,\lambda)$ avec $\lambda$ la mesure de Lebesgue. 
% \begin{enumerate}
% \item Supposons que $g$ appartient à $\mathcal C^\infty_c(\R)$. Montrer que $f\star g$ est de classe $\mathcal C^\infty$ sur $\R$.
% \item  Soit $\delta \in \mathcal C^\infty_c(\R,\R_+)$ telle que $\int_\R \delta(x)\,dx=1$. Étant donné $\varepsilon>0$, on pose, pour tout $x\in \R$,
% $$
% \delta_\varepsilon(x)=\frac{1}{\varepsilon}\delta\left(\frac{x}{\varepsilon}\right).
% $$
% \begin{enumerate}
% \item Montrer que pour tout $\eta>0$, $\lim_{\varepsilon\to 0}\int_{|x|>\eta}\delta_\varepsilon(x)\,dx=0$.
% \item Soit $f\in \mathcal C_c(\R)$. Montrer que $f\star \delta_\varepsilon\in C^{\infty}_c(\R)$ et que $\lim_{\varepsilon \to 0}\|f\star \delta_\varepsilon-f\|_1=0$.

% {\em Indication : on pourra utiliser l'uniforme continuité de la fonction $f$.}
% \item En déduire que $\mathcal C_c^\infty$ est dense dans $L^1(\R,\lambda)$.
% \end{enumerate}
% \end{enumerate}
% }

\end{document}